\documentclass{cernatsnote}
\usepackage{siunitx}
\usepackage[table,xcdraw]{xcolor}
\usepackage{soul}
%\usepackage[colorinlistoftodos]{todonotes}
\usepackage{placeins}
\usepackage{titlesec}
\usepackage{subfigure}
\usepackage{float}
\usepackage{booktabs}
\setcounter{secnumdepth}{4}
\setcounter{tocdepth}{4}    %
\usepackage{amssymb}
\usepackage{url}

\usepackage{booktabs}
\usepackage{multirow}
\usepackage{rotating,tabularx}
\usepackage{tikz}
\usetikzlibrary{arrows}
\usetikzlibrary{patterns}
\usepackage{array}
\usepackage{mathtools}
\usepackage{makecell}

\usepackage{multirow}
\usepackage{graphicx}
\usepackage{threeparttable}

\newcommand\ytl[2]{%
  \parbox[b]{8em}{\hfill{\color{black}#1}~$\cdots\cdots$~}%
  \makebox[0pt][c]{$\bullet$}\vrule\quad%
  \parbox[c]{8.5cm}{\vspace{7pt}\color{black}\raggedright #2.\\[7pt]}\\[-3pt]%
}

\title{TN1 Technical note on solution implemented for the primary beam energy reduction}
\author{
	E. Johnson, A. Waets, P.A. Arrutia Sota, K. Bilko, M. Delrieux,\\ N. Emriskova, A. Coronetti,  M.A. Fraser, R. Garcia Alia\; \\		
	CERN, CH-1211 Geneva, Switzerland
}
\emailb{ruben.garcia.alia@cern.ch}
\emaila{eliott.philippe.johnson@cern.ch}
\date{\today}

\setlength {\marginparwidth }{2cm}
\begin{document}
\maketitle

\begin{abstract}
This technical note describes a solution implemented for the primary beam energy adjustment for the heavy ion irradiation activities in the CHARM facility, as part of the CHIMERA activity at CERN. The beam energy was varied by scaling the magnetic field in the Proton Synchrotron (PS) during acceleration and transport. The PS's magnet strengths were adjusted by an algorithm (makerule) that computes the magnetic field required for a given beam energy, allowing automatic scaling of magnets in the F61 and T8 transfer lines. This note presents the results of the beam energy measurements from the November 2022 run and explores potential areas for system improvement.
\end{abstract}
\\ \\ \\ 

\newpage

\sethlcolor{yellow}
\begin{table}[!htp]
\centering
{\tiny
\begin{tabular}{@{}p{2.5cm}p{2cm}p{2cm}p{2.4cm}p{5cm}@{}}
\toprule
Contract name & Contract number - ESA & Contract number - CERN & Task & Technical Note or Report \\
\midrule
Development of High Energy Beam (range and LET) for Radiation Tests of Highly Integrated Electronics components &
4000134554\linebreak/21/NL/KML/rk &
KM5174\linebreak/KT/SY/273A &
Task 1: CERN CHARM ion beam energy reduction &
TN1: Technical note on solution implemented for the primary beam energy reduction\\
&  &  & Task 2: Beam calibration and dosimetry & \hl{TR1: Test Report with calibration and dosimetry measurements}\\
 &  &  &  & TN2: Technical Note on the dosimetry methodologies and procedures for calibration and routine beam operations\\
High Energy Beam Intensity Adjustment for SEE tests of COTS EEE Components & 4000136601\linebreak/21/NL/KML/rk & KM5450\linebreak/KT/SY/276R & Task 1: Primary beam intensity adjustments & TN1: Technical Note on solution implemented for the flux adjustments\\
 &  &  & Task 2: Beam calibration and dosimetry & TR1: Test report with initial and final results of calibration and dosimetry measurements, example of test report that will be provided for routine operation\\
 &  &  & Task 3: Preparation of test execution process for industry & TN2: Technical note. Guideline for test users: with necessary information for external users to prepare and execute a test at the facility\\
 &  &  &  & TR2: Test report of SEE tests\\
 \bottomrule
\end{tabular}
}
\end{table}

\newpage

\begingroup
\color{black}
\tableofcontents
\endgroup

\pagebreak

\section{History of Changes}

\noindent
\begin{tabularx}{\textwidth}{|c|c|X|}
\hline
\textbf{Version} & \textbf{Date} & \textbf{Changes} \\
\hline
1.0 & 24-03-2023 & Initial document creation. \\
\hline
1.1 & 15-05-2023 &  Added information on unexpected energy change. Added timeline with different milestones and MDs.\\
\hline
1.2 & 13-06-2023 & Added information to the conclusion. \\
\hline
1.3 & 28-08-2023 & Changed SoW document number to ESA and CERN contract number. Removed ESA logo. Moved section on FLUKA simulation from ETR1 to ETN1.\\
\hline
1.4 & 11-09-2023 & Merged conclusion of ETR1 to ETN1.\\
\hline
\end{tabularx}

\newpage

\section{Introduction}
In the field of radiation testing of electronic components, controlling the primary beam energy is a critical aspect of ensuring accurate and reliable results. This is particularly significant when working with a single ion or a few ions. Changing the energy across a relatively broad value range is crucial to achieving substantial variability in Linear Energy Transfer (LET), a measure of the energy deposited by a particle as it travels through a material. The primary beam energy determines the LET, which in turn influences the type and amount of radiation-induced effects on the electronic components being tested. By varying the primary beam energy, different ranges of LET become accessible, allowing for the study of radiation effects at various levels of energy deposition. This is important for understanding how electronic components will behave in diverse radiation environments. For instance, in space applications, electronic components are exposed to high-energy particles with a wide range of LET values. Therefore, it is essential to understand how these particles will affect the component's performance and reliability. The ability to control the primary beam energy and explore different LET values is critical for accurately characterizing the radiation response of electronic components and designing robust systems for space and other radiation environments.

This technical note detailing the solution implemented for primary beam energy variation is an important milestone to providing a heavy ion irradiation program at CERN because it provides a method for controlling primary beam energy and exploring a wide range of LET values of which the achieved results have been compared to the key technical requirements listed in Table \ref{tab:technical_requirements}. The variation of the primary beam energy was achieved by scaling the magnetic field in the Proton Synchrotron (PS) during acceleration by adjusting the PS's magnet strengths according to a makerule that computed the magnetic field required for a given beam energy. The makerule also allows for the automatic scaling of magnets in F61 and the T8 transfer line, which are required to transport the heavy ion beam to CHARM. The note details the various steps taken during the run, including characterization of the three beam energies, testing of the four Device Under Test (DUT) setups, and a degrader scan. Furthermore, the note presents the results of the beam energy measurements during the November 2022 run and of the FLUKA simulation.

\begin{figure}[!htb]
\centering
\includegraphics[width=1.0\textwidth]{images/CCC_eastt8_small.png}
\caption{The CERN accelerator complex. The path of the lead ion beam used for the ion irradiation activity in CHARM is highlighted in color.}
\label{fig:CCC}
\end{figure}

\begin{table}[!htp]
\centering
\begin{threeparttable}
\caption{Summary of key technical requirements}
\label{tab:technical_requirements}
\begin{tabular}{@{}p{3.5cm}p{3cm}p{3.5cm}p{5cm}@{}}
\toprule
    & Requirement & Achieved\\
\midrule
LET MeVcm$^2$/mg & 10 to 30 & 13.4 to 26.5\tnote{$\dagger$} \cite{waets_tr1_2023}\\\bottomrule
\end{tabular}
\begin{tablenotes}
\item[$\dagger$] The range noted pertains to standard runs. In runs with degraders and energy scans, the maximum LET was significantly higher and need to be quantified in future simulations.
\end{tablenotes}
\end{threeparttable}
\end{table}

\section{Energy variation}

The Proton Synchrotron (PS) is a crucial component of the accelerator complex, responsible for accelerating particles from the Proton Synchrotron Booster (PSB) to the Super Proton Synchrotron (SPS) and ultimately to the Large Hadron Collider (LHC). Additionally, the PS is capable of receiving and accelerating heavy ions from the Low Energy Ion Ring (LEIR) for extraction to the East Area, making it a versatile machine. RF cavities are used to accelerate charged particles, such as protons or lead ions, to high energies, while 100 dipoles bend the beam around the PS's 628 m circumference \cite{gilardoni_fifty_2011}. The particles are then extracted using slow extraction \cite{fraser_feasibility_2022} through a beam line and transported to the CHARM facility, which houses the beam instruments and irradiation setups.
\\

The beam energy refers to the kinetic energy per nucleon $E_{kin}$ of the charged particle beam. For an ion beam, the total kinetic energy, denoted $E_{kin, TOT}$, can be obtained by multiplying $E_{kin}$ by the atomic mass number $A$ (where $A = Z + N$, representing the total number of protons $Z$ and neutrons $N$ in the nucleus). The total momentum for an ion beam is given by \cite{chao_handbook_2013}

$$pc={E_{0}\sqrt{\gamma^{2}-1}}$$

$$pc = E_{0}\sqrt{\left [ \left( \frac{E_{kin, TOT}}{E_{0}}+1\right )^{2}-1\right ]}$$
where, $E_{0}$ is the rest mass of the ion and $\gamma=\frac{E}{E_{0}}=\frac{E_{0}+E_{kin}}{E_{0}} = \frac{E_{kin}}{E_{0}}+1$
\\

The ion irradiation activity in CHARM (CHIMERA) uses a lead ion beam. In the PS the beam is a partially stripped ion beam of Pb$^{54+}$, which means that 28 $e^{-}$ remain on the ion for the isotope A=208.
\\

\begin{figure}
    \centering
    \begin{minipage}{0.45\textwidth}
        \centering
        \includegraphics[width=1.0\textwidth]{images/PS_BEAM_ENERGY/vaccum_window.jpg}
        \caption{Vacuum window connecting the PS to the transfer line leading to the East Area.}
        \label{fig:vaccum window}
    \end{minipage}\hfill
    \begin{minipage}{0.45\textwidth}
        \centering
        \includegraphics[width=1.0\textwidth]{images/PS_BEAM_ENERGY/stripping.jpg} 
        \caption{Diagram illustrating the stripping process, where an ion loses its remaining electrons. The resulting ion has a higher charge state when it travels through the transfer line.}
        \label{fig:stripping}
    \end{minipage}
\end{figure}

As an example, for a kinetic energy of 1 GeV/$u$ (per nucleon) the total momentum would be:

$$p = E_{0}\sqrt{\left [ \left( \frac{1\text{ [GeV]}\cdot 208}{E_{0}}+1\right )^{2}-1\right ]} = 351.9 \left[\frac{\text{GeV}}{\text{c}}\right]$$
where the rest mass $E_{0}$ of Pb$^{54+}$ is:

$$E_{0} = m_{Pb54+}= 82\cdot m_{proton} + 126\cdot m_{neutron} + 28\cdot m_{e^{-}} - m_{defect} = 193.74 \left[\frac{\text{GeV}}{\text{c}^{2}}\right]$$
with mass defect; see Table \ref{table:masses}: $m_{defect}=82\cdot m_{proton} + 126\cdot m_{neutron} + 28\cdot m_{e^{-}} - 208\cdot m_{u}$ 

\begin{table}[h!]
\centering
\caption{Rest masses \cite{boston_university_nuclear_nodate}.}
\begin{threeparttable}
\begin{tabular}{lr}
\toprule
Particle & mass GeV/$\text{c}^{2}$\\
\midrule
$m_{proton}$ & 0.93828      \\
$m_{neutron}$ & 0.93957      \\
$m_{e^{-}}$ & 0.000511 \\
$m_{u}$ & 0.9315       \\
\bottomrule
\end{tabular}
\label{table:masses}
\end{threeparttable}
\end{table}

The beam momentum is related to the magnetic rigidity [Tesla meter] as follows: 
$$B\rho \text{ [T m]} = 3.3356\cdot p/q \text{ [GeV/c]}$$
where for the PS, $\rho = 70.0789$ m. This formula is used to calculate the bending B-field in the PS dipoles.
\\

For the 2022 CHIMERA run, three primary kinetic energies at flat-top in the PS were selected: 1000, 750, and 650 MeV/$u$, and the associated required magnetic fields (B-fields) were calculated, as shown in Fig. \ref{fig:lookup table} and Table \ref{tab:KE_table}. However, interaction with vacuum windows, instruments, and air molecules during transport through the transfer line lowers the beam's kinetic energy. Therefore, a FLUKA model simulation is required to determine the beam energy at the Device Under Test (DUT) in CHARM. The FLUKA simulation model includes rigidities for each of the primary energies, along with an accurate representation of the material budget that the beam crosses when transported through the transfer line, which includes all vacuum windows, beam instrumentation, and air regions. The simulations reveal that the compound effect of all the materials in the beam causes energy straggling, resulting in significantly lower beam kinetic energies at the DUT position. Table \ref{tab:KE_table} presents the primary kinetic energy degradation from the PS to the DUT. Consequently, the range of beam LETs, which is one of the fundamental radiation effects testing metrics, that can be obtained by varying the beam energy is larger than the LET range at flat-top in the PS. The simulations show that by varying the primary beam energy extracted from the PS, the resulting LET at the DUT can vary between 10 and 30 MeVcm$^2$/mg. When additional dedicated sheets of material, such as PPMA, are placed in the beam, it is possible to achieve LETs greater than 30 MeVcm$^2$/mg.
\\

Significant effort has been invested in automating the scaling of magnets in F61 and the T8 transfer line to reduce the operational overhead of operating the irradiation facility at different beam energies. The removal from operation of the Pole Face Windings (PFW) in the PS was necessary to allow for continuous energy variation as they do not scale with beam rigidity in a linear way. The T8 transfer line has been converted to Pulse to Pulse Modulation (PPM), which means that the magnet strengths are user-based and different users (or energies) can be sent to the East Area in parallel. This enables the use of different beams during the same super-cycle, whereas in non-PPM mode, only a single user or type of beam could be used. A makerule is used to compute the correct strength to apply based on the PS B-field (and thus beam energy).
\\

Moreover, since the ion beam is partially stripped in the PS and becomes fully stripped after passing through the exit vacuum window in the F61 transfer line, located after the first quadrupole in the transfer line (Q74L), different rigidities are required for acceleration and transport, respectively. Therefore, the magnets in the transfer line must be pulsed with a rigidity matched to the fully stripped ion beam. The makerule computes a partially stripped momentum for the first quadrupole in F61 and a second momentum for a fully stripped ion beam for the rest of the transfer line down to CHARM. This automatic scaling is highly advantageous, as energy scans can be performed by merely adjusting the B-field at the flat-top of the PS.
\\

Nevertheless, there are certain limitations in the automatic scaling process that may affect operations. For example, the extraction path occasionally requires manual corrections. This process involves adjusting the strength of the septum magnets used for extraction, notably SMH57 and SMH61. Although no corrections were made to these septums during the run, small errors in trim could potentially lead to transmission issues in the future. Additionally, the scaling of the dipole to the East Dump line, which cannot be associated with a makerule, had a minor impact during Machine Developments (MDs). Any necessary energy adjustment required manual scaling of the magnet current. Looking ahead, the development of automated solutions to address these limitations could enhance MD efficiency.


\begin{figure}[!htb]
\centering
\includegraphics[width=1.0\textwidth]{images/PS_BEAM_ENERGY/kinetic_energy_lookup_chimera.png}
\caption{Lookup table for the lead ion beam flat-top in the Proton Synchrotron (PS). The magnetic field required for acceleration to three different energies (1000, 750, and 650 MeV/$u$) as well as energy scan measurements to the East Dump is shown.}
\label{fig:lookup table}
\end{figure}

\begin{table}[!htbp]
\centering
\caption{Table showing the kinetic energy and LETs used during the ESA November 2022 run at flat-top in the PS and at the DUT. The term 'USER' refers to the settings the PS is loaded with. A USER profile saves all magnet strengths, which can be easily accessed and applied to the PS when that particular USER is called, similar to a preset configuration.}
\begin{threeparttable}
\label{tab:KE_table}
\begin{tabular}{@{}m{1.6cm}m{1.6cm}m{2.8cm}m{2.8cm}m{4cm}@{}}
\toprule
$E^{PS}_{kin}$ {[}MeV/$u${]} & $E^{DUT}_{kin}$\tnote{$\dagger$} [MeV/$u$] & $LET^{PS}$\tnote{$\dagger$} [MeV$\cdot$~cm$^2$/mg] & $LET^{DUT}$\tnote{$\dagger$} [MeV$\cdot$~cm$^2$/mg] & USER \\ \midrule
1000 & 605 & 11.7 & 13.4 & CPS.USER.EAST4 \\
750 & 300 & 12.6 & 18.4 & CPS.USER.EAST3 \\
650 & 140 & 13.2 & 26.5 & CPS.USER.MD5 \\ \bottomrule
\end{tabular}
\begin{tablenotes}
\item[$\dagger$] FLUKA simulation \cite{waets_tr1_2023}.
\end{tablenotes}
\end{threeparttable}
\end{table}


Figure \ref{fig:bfield} illustrates the magnetic field profile of the main dipoles in the Proton Synchrotron (PS) produced by the main magnet power supply (POPS). The profile includes two distinct regions - the injection plateau where ions are injected from LEIR, and the flat-top extraction plateau where ions have achieved maximum energy and are ready for extraction. Note that the spike at the end, which was introduced to overcome certain limitations of POPS, is irrelevant to the extraction process and can be disregarded.

\begin{figure}[!htb]
\centering
\includegraphics[width=0.6\textwidth]{images/PS_BEAM_ENERGY/average_b_field_chimera.png}
\caption{CHIMERA's magnetic field at different energies, with injection plateau at 684 G and extraction plateau at varying B-field strengths corresponding to different energy levels.}
\label{fig:bfield}
\end{figure}

During the CHIMERA ion run in November 2022, the experiment lasted for 5 days, from the afternoon of Wednesday 23 November to the  morning of Monday 28 November. The energies used during the run were 1000, 750, and 650 MeV/$u$ and were each used for almost the same amount of time. The 650 MeV/$u$ beam was used for 36\% of the time, the 750 MeV/$u$ beam was used for 28\%, and the 1000 MeV/$u$ beam was used for 35\%. The run was divided into specific tasks and timeframes, see Fig. \ref{fig:timestamp_energies}. Wednesday and Thursday (16th and 17th) were used for beam preparation, Friday and Saturday (25th and 26th) were devoted to ESA's experiments, and Sunday (27th) was used for backup or for additional machine development. During the first few hours of the beam time, the focus was on characterizing the three energies and varying the intensity using the standard beam instruments. The next few hours were dedicated to testing all four DUTs (Canberra Si diode and three Static random-access memory SRAMs) sequentially for all energies, followed by a run overnight to accumulate statistics on the Renesas memory \cite{noauthor_rmlv0816bgsa_nodate}. On Thursday morning, the team changed the degrader thickness twice. Friday and part of Saturday were dedicated to ESA's experiment. The rest of Saturday night and Sunday were left for instrument and optics measurements for CHIMERA.

\begin{figure}[!htb]
\centering
\includegraphics[width=1.0\textwidth]{images/PS_BEAM_ENERGY/user_mapping_timestamp.png}
\caption{Timestamps of beam energies: the plot displays the time stamps associated with different beam energies used during the November run.}
\label{fig:timestamp_energies}
\end{figure}

Following the ESA run, an automatic script was used to ramp the beam energy for further development purposes. A dedicated energy scan was conducted on the diode using 5 different beam energies, ranging from 775 MeV/$u$ to 900 MeV/$u$. A separate experiment was carried out to measure the lowest beam energy from the extraction of the PS that could be recorded at the East Dump, which was found to be 283 MeV/$u$. The main limitation was that the bending magnet to the dump is not scaled with momentum, and hand corrections were needed to propagate the beam to the dump (which could be scaled and set by a script), as well as the beam spot on the BTV being large and weak at this energy, which made observation difficult. Lower energies could be tried out in the future.

\begin{figure}[!htb]
    \centering
    \begin{minipage}{0.45\textwidth}
        \centering
        \includegraphics[width=1.0\textwidth]{images/PS_BEAM_ENERGY/energy_scan_chimera 1.png}
        \caption{Plot of the B-field showing different flat-tops during an energy scan.}
        \label{fig:energy_scan}
    \end{minipage}\hfill
    \begin{minipage}{0.45\textwidth}
        \centering
        \includegraphics[width=1.0\textwidth]{images/PS_BEAM_ENERGY/energy_scan_timestamp_chimera 1.png} 
        \caption{Energy scan with timestamp.}
        \label{fig:energy_scan_timestamp}
    \end{minipage}
\end{figure}

During the ESA run, there were two instances where the displayed energy did not match the selected energy. This discrepancy occurred not because of a required energy change, but due to an unwanted swap of the supercycle. Multiple supercycles, approximately 5-10, are preprogrammed in anticipation of the day's needs and can be swiftly swapped based on the demands of the SPS and LHC. CHIMERA operated with three different cycles, each corresponding to a particular energy. With this system, whenever an energy change is required, the operator in the CCC must manually swap the ion cycle to match the correct energy in the current supercycle. It is also considered good practice to manually change the stack of preprogrammed supercycles to reflect the newly selected energy. However, this process is time-consuming and error-prone, especially if energy changes are frequent. The issue arose when a supercycle swap was made, and the operator either lacked the time or forgot to replace the ion cycles with the appropriate energy. Consequently, for a few spills, the incorrect energy was played. Although the issue was quickly identified and rectified both times, it's clear that an improvement is needed. One solution is to have a single ion cycle and implement an automatic script to adjust the energy. This is the preferred solution and will be implemented for the next ion run. For sensitive DUTs, an additional interlock that blocks any beam not at the correct energy could be implemented, provided it aligns with the given requirements.

\section{FLUKA Simulation}
Determining the LET at the test location is essential and is a function of both particle type and energy. The T08 transfer line which transports the beam from the PS to CHARM features \textbf{a considerable amount of in-beam instrumentation} and monitoring units, along a large fraction of distance traveled by the \textbf{beam in air} (delimited by vacuum windows). The beam line is predominantly used for 24 GeV/c proton operation for which the compound material budget has very little effect on the beam quality for protons when delivered to the test facilities IRRAD and CHARM. The situation changes drastically when the transport of VHE ions through T08 is requested, for which the present material budget in the different non-vacuum regions affects the beam quality through:
\begin{itemize}
    \item \textbf{Electronic stopping power} (dE/dx): Interaction of the beam particle charge with the electronic structure inside material results in a loss of particle kinetic energy, generally quantified by the Bethe formula \cite{Bethe30}, according to which, the average stopping power is proportional to the ratio of the target atomic and mass numbers Z$_T$ and A$_T$ and the density $\rho_T$:
\begin{equation}\label{stopping}
    dE/dx \propto \frac{Z_T}{A_T}\rho_T;
    \end{equation}
    Particles sourced with a fixed energy are subject to energy straggling, i.e. fluctuations in energy loss caused by the statistical nature of energy loss processes for charged particles in matter. A spread-out energy distribution around the mean given by the Bethe formula is inevitable after traversing material.
    \item \textbf{Inelastic interactions}: The beam particles have sufficient energy for triggering nuclear reactions with any material it crosses \cite{Lannunziata16}. The macroscopic cross section is proportional to the target material density $\rho_T$:

    \begin{equation}
    \sigma \propto \rho_T,
    \end{equation}
The microscopic cross section $\sigma$ for a reaction between a projectile and target can be approximated in a geometrical sense to first-order approximation \cite{Krane88}:  $\sigma = \pi (R_T + R_P)^2$   with the nuclear radius given by $R \propto A^{1/3}$.
    Evidently, when the beam crosses e.g. vacuum windows, a primary particle can be lost in a nuclear reaction and only secondary fragments remain. Due to the high beam energy and by consequence high velocity they will be transported predominantly in a forward fashion: this can cause considerable beam contamination of particles with different charge and energy; and by consequence different LET.
\end{itemize}

Both processes also affect the fluence: collisions with the electronic structure of materials causes angular scattering which spreads out the beam over larger surface areas; inelastic interactions deteriorate the amount of primary particles in the beam. The discussion of these effects are however out of the scope of this report.\\
\\
A suitable measure of the extent to which these two physics processes affect the beam characteristics is required. In the energy range under consideration, the stopping power exhibits a weak material dependency (due to the proportionality to Z/A). Hence, the material density and cumulative distance traveled through the material are the dominating factors. An effective, descriptive material property is the \emph{surface density} $l'$ which is the product of the material length $l$ and material density $\rho$:
\begin{equation}
l' = l \times \rho
\end{equation}
in units of [g/cm$^2$] and is commonly used when quantifying particle penetration in matter \cite{Lechner18}. We refer to the material or region length as the distance traveled within the material by an ideal beam particle. The surface density is also useful to describe the prevalence of inelastic collisions inside materials.\\
\\
The mean free path of a given type of particle and corresponding energy inside a material is also called the \emph{inelastic scattering length} and is a function of the projectile type, projectile energy and target material properties such as atomic composition and density. To give an example in the scope of the T08 simulations: the inelastic scattering length for Pb ions extracted at 1 GeV/n in titanium is roughly 4 cm, whereas in air it is 60 m. These values are similar for lower extracted energies (down to 100s MeV/n) and were all taken from the standard FLUKA simulation output file.\\
\\
 The surface density values for all (non-vacuum) materials the beam sees along the transfer line can already shed some light into where primary beam energy straggling and fragmentation is prevalent. In Table \ref{tab:material}, all non-vacuum materials seen by the beam are listed, including the surface density in the final column. To highlight an example: the XION070 instrument features two stainless steel vacuum windows and a set of stainless steel foils. Each of these, despite their $\mu$m level thickness, has a comparable surface density as the argon gas that the instrument is filled with which has a longer length but a much lower density. One can also infer from the table that the \textbf{biggest effects can be expected in the different air sections} given that these reach the highest surface density values due to the long distance crossed by the beam. The cumulative surface density is 5.38 g/cm$^2$, of which 3.88  g/cm$^2$, i.e., 72\% is due to air. \\
 \\
 However, a more detailed measure can be made by \textbf{Monte Carlo (MC) calculations}. MC simulations are an essential tool to quantify the effects of transport along the beam line. They are a decisive tool when it comes to providing a more fundamental and in-depth description of the beam and radiation environment and also allow to run multiple configurations and related optimizations in parallel; both of which are often not available experimentally. The general purpose Monte Carlo code FLUKA \cite{Battistoni15, FlukaWeb} is a most suitable tool, regularly used in accelerator environments and extensively benchmarked on a microscopic level \cite{Ahdida22, Braun14, Battistoni07}. When discussing the different elements in the form of their FLUKA simulation model analog we will generally talk about \emph{regions}, i.e., segments of the geometry composed of a single material. All calculations in this study were carried out in FLUKA versions 4.2 and 4.3 with checks between the two versions for consistency.
 
\begin{table}[htp]
\scriptsize
\centering
\caption{Material budget table for the November 2022 T08 beam line configuration and the corresponding FLUKA simulation model. Only the non-vacuum regions and their parent elements are listed, as function of their distance from the DUT location (along the $s$-axis). The region thickness is expressed in [cm] except for very thin subcomponents of instruments such as electrode foils or vacuum covers for which the thickness is expressed in [$\mu$m]. The surface density expression in units of [0.1 g/cm$^2$] was chosen to ease the comparison between different regions. In addition, the cumulative non-vacuum length is quoted.}
\label{tab:material}
\begin{threeparttable}
\resizebox{\textwidth}{!}{%
\begin{tabular}{lllllllS}
\toprule
\toprule
 & & Element & Position  & Thickness & Material & Density & \multicolumn{1}{r}{Surface density}\\
 & & & [m] & [cm] &  & [g/cm$^3$] & [0.1~g/cm$^2$]  \\
 \toprule
 & & SOURCE (PS) \\
 \midrule
 \multirow{10}{*}{\rotatebox[origin=c]{90}{F61} $\begin{dcases*} \\ \\ \\ \\ \\ \\ \\ \\ \\ \end{dcases*}$} & 1. & PS vacuum window & 123.97 & 0.01 & Ti & 4.540 & .45 \\
 & & \multicolumn{1}{r}{Air section} & 123.97 & 30.0 & air & 1.293$\cdot$10$^{-3}$ & .39\\
 & 2. & BTV01 vacuum window & 123.25 & 0.01 & Ti & 4.540 & .45 \\
 & 3. & BCTF022 vacuum window & 113.37 & 0.02 & Al & 2.699 & .54\\ 
 & & \multicolumn{1}{r}{Air section} & 113.37 & 17.7 & air & 1.293$\cdot$10$^{-3}$ & .23\\
 & 4. & BCGAA23 & 113.09 & $2\times 50 \mu$m & Al & 2.699 & .27 \\
 & & (Gaseous scint.) & & 6 & N$_2$ & 1.17$\cdot$10$^{-3}$ & .07 \\
 & & \multicolumn{1}{r}{Air section} & 113.09 & 4.0 & air & 1.293$\cdot$10$^{-3}$ & .05\\
 & 5. & XSEC023 & 112.81 & $2\times25 \mu$m & stainless steel & 7.9 & .40\\
 & & & & $2\times5\times5 \mu$m & Al & 2.699 & .13 \\
 & & \multicolumn{1}{r}{Air section} & 109.89 & 250.0 & air & 1.293$\cdot$10$^{-3}$ & 3.23\\
 & 6. & T08 vacuum window & 107.30 & 0.02 & Ti & 4.540 & .91 \\
 \\
 \multirow{9}{*}{\rotatebox[origin=c]{90}{T08} $\begin{dcases*} \\ \\ \\ \\ \\ \\ \\ \end{dcases*}$} & 7. & T08 vacuum window & 34.0 & 0.02 & Al & 2.699 & 0.54\\
 & & \multicolumn{1}{r}{Air section} & 34.00 & 15.0 & air & 1.293$\cdot$10$^{-3}$ & 0.19\\
 & 8. & XSEC070 & 33.80 & $2\times25 \mu$m & stainless steel & 7.9 & 0.40\\
 & & & & $21\times5 \mu$m & Al & 2.699 & 0.28\\
 & & \multicolumn{1}{r}{Air section} & 33.56 & 10.0 & air & 1.293$\cdot$10$^{-3}$ & 0.13\\
 & 9. & XION071 & 33.46 & $2\times25 \mu$m & stainless steel & 7.9 & 0.40\\
 & & & & $21\times5 \mu$m & Al & 2.699 & 0.28 \\
 & & & & 24.0 & Ar & 1.66$\cdot$10$^{-3}$ & 0.40 \\
 & & \multicolumn{1}{r}{Air section} & 33.22 & 75 & air & 1.293$\cdot$10$^{-3}$ & 0.97\\
 \\
 \multirow{16}{*}{\rotatebox[origin=c]{90}{IRRAD} $\begin{dcases*} \\ \\ \\ \\ \\ \\ \\ \\ \\ \\ \\ \\ \\ \end{dcases*}$} & 10. & BPM1 & 32.48 & 0.1 & Cu & 8.96 & 0.90\\
 & & & & 0.069 & Kapton & 1.42 & 0.98\\
 & 11. & IRRAD vacuum window & 32.48 & 0.02 & Al & 2.699 & 0.54\\
 & 12. & IRRAD vacuum window & 28.98 & 0.02 & Al & 2.699 & 0.54\\
 & & \multicolumn{1}{r}{Air section} & 28.98 & 411.0 & air & 1.293$\cdot$10$^{-3}$ & 5.31\\
 & 13. & BPM2 & 24.87 & 0.1 & Cu & 8.96 & 0.90\\
 & & & & 0.069 & Kapton & 1.42 & 0.98\\
 & & \multicolumn{1}{r}{Air section} & 24.87 & 174.0 & air & 1.293$\cdot$10$^{-3}$ & 2.25\\
 & 14. & XSCI & 23.13 & 0.028 & plastic scint. & 1.032 & 0.29\\
 & & \multicolumn{1}{r}{Air section} & 23.1 & 556.0 & air & 1.293$\cdot$10$^{-3}$ & 7.19\\
 & 15. & BPM3 & 17.54 & 0.1 & Cu & 8.96 & 0.90\\
 & & & & 0.069 & Kapton & 1.42 & 0.98\\
 & & \multicolumn{1}{r}{Air section} & 17.54 & 483.0 & air & 1.293$\cdot$10$^{-3}$ & 6.25\\
 & 16 & BPM4 & 12.71 & 0.1 & Cu & 8.96 & 0.90\\
 & & & & 0.069 & Kapton & 1.42 & 0.98\\
 & 17. & IRRAD vacuum window & 12.71 & 0.02 & Al & 2.699 & 0.54 \\
 & 18. & IRRAD vacuum window & 10.00 & 0.02 & Al & 2.699 & 0.54\\
 \\
 \multirow{8}{*}{\rotatebox[origin=c]{90}{CHARM} $\begin{dcases*} \\ \\ \\ \\ \\ \\ \end{dcases*}$}& & \multicolumn{1}{r}{Air section} & 10.00 & 8.0 & air & 1.293$\cdot$10$^{-3}$ & 0.10\\
 & 19. & XION094 & 9.92 & $2\times25 \mu$m & stainless steel & 7.9 & 0.40 \\
 & & & & $21\times5 \mu$m & Al & 2.699 & 0.28\\
 & & & & 24.0 & Ar & 1.66$\cdot$10$^{-3}$ & 0.40\\
 & & \multicolumn{1}{r}{Air section} & 9.68 & 10.0 & air & 1.293$\cdot$10$^{-3}$ & 0.13\\
 & 20. & XSEC094 & 9.58 & $2\times25 \mu$m & stainless steel & 7.9 & 0.40\\
 & & & & $5\times5 \mu$m & Al & 2.699 & 0.07\\
 & & \multicolumn{1}{r}{Air section} & 9.34 & 934.0 & air & 1.293$\cdot$10$^{-3}$ & 12.08\\
 \bottomrule
 & & DUT (CHARM) \\
\bottomrule
& & \textbf{Cumulative:} & & \textbf{30.83 m} & & & \multicolumn{1}{l}{\textbf{5.36 g/cm$^2$}}\\
\bottomrule
\bottomrule
\end{tabular}
}
\end{threeparttable}
\end{table}
%All:  7.12 + 3.59 + 29.17 + 13.68 = 53.56
%Air: 38.76
\subsection{Geometric model of the T08 beam line}


\begin{figure}[t]
    \centering
    \includegraphics[width=\textwidth]{images/schematic.png}
    \vspace*{-0.5cm}
    \caption{Schematic representation of the T08 beam line, showing the main elements between the PS extraction point and test location in CHARM. The distance from the test location is measured along the $s$-axis of the beam reference frame. The beam line is subidivided in different zones: particles are extracted from the PS into the F61 before transported through the T08 line to experimental facilities IRRAD and ultimately CHARM. Optics elements such as bending dipole magnets (dark red) and quadrupole magnets (purple) are shown, along with the sections where the beam travels in air (cyan). The in-beam instrumentation units and vacuum windows are indicated and numbered according to Table \ref{tab:material}.}
    \label{fig:schematic}
\end{figure}

\begin{figure}[b]
    \centering
    \includegraphics[width=\textwidth]{images/FLUKA_geometry.png}
    \vspace*{-0.5cm}
    \caption{Snapshot of the FLUKA geometry in the $xz$-plane as visualised in Flair \cite{Vlachoudis09}.}
    \label{fig:snapshot}
\end{figure}

In this report, the T08 beam line is described from the extraction point in the PS down to the \emph{device-under-test} (DUT) location in CHARM. By accelerator physics convention, we define the reference frame of the beam using the $s$-coordinate, which follows the beam along its ideal trajectory through the transfer line; the associated transverse coordinates $x$ and $y$ are of little relevance in this report. A schematic representation of the beam line is shown in Fig. \ref{fig:schematic}. The geometric model origin was chosen to be the DUT location in CHARM; its orientation was chosen such that the beam frame coincides with the geometric model frame in IRRAD and CHARM. The geometric model frame is spanned by the Cartesian $x$-, $y$- and $z$ axes, a snapshot of this geometry is shown in Fig. \ref{fig:snapshot}. \\
\\
As stated above, the goal of this study is to determine primary beam characteristics like the beam energy and associated LET. We are therefore only interested in accurately modeling direct interactions of the beam with the beam line infrastructure. The geometry was constructed using the LineBuilder python code \cite{Mereghetti12}, relying on the optics settings used during operation, dimensions from technical drawings and in-person inspections. The resulting model spans $\sim 140$ m from the PS extraction point to the chosen device-under-test (DUT) position located in the CHARM facility. The FLUKA model incorporates the correct materials, thicknesses (down to $\mu$m level) and respective distances for all elements the beam travels through. A complete description of the material budget seen by the beam is detailed in Table \ref{tab:material}, as it is implemented in the FLUKA model. We however restrict this detailed representation to the beam line aperture; outside, regions were assigned the FLUKA "black hole" material where particle tracking is killed upon entrance. This is also a necessary measure to keep the simulation time within limits, as will be further detailed in Section \ref{sec:physics} below.

\subsection{Physics and scoring settings}\label{sec:physics}
The particle source used in the FLUKA simulations was implemented as accurately as possible. The beam emittance and momentum spread were determined by measurements, serving as input to generate a phase space distribution with $x$, $x' (=dx/ds)$, $y$, $y'$ and $dp/p$ using \emph{pyBT}, a python package used by the Accelerator Beam Transfer group at CERN. The momentum dispersion  at extraction $dp/p_0$ is Gaussian shaped with a standard deviation that generally amounts to $\pm 0.3\%$. For each extracted kinetic energy, the $y$ and $y'$ were scaled by the relativistic factor $\beta \gamma$ to account the the fact that the beam gets larger when decelerated. From the generated initial distribution of 1M particles, 100K were randomly sampled to be simulated in FLUKA.\\
\\
The user-defined FLUKA simulation settings were chosen as a compromise between accuracy and simulation time: 
full transport of heavy ions (without any form of biasing) and hadronic particles down to 1 keV is necessary to track secondary fragments, neutrons are even transported down to 10$^{-5}$ eV which is the lowest limit. The generation of those fragments is accounted for by implying the appropriate heavy ion physics interaction models (such as electromagnetic dissociation). All other physics is properly handled by using the \emph{PRECISIOn} physics defaults. The interaction of heavy, charged particles with matter causes the production of particles in the electromagnetic sector (photons, electrons, positrons) through Bremsstrahlung and pair production processes. These particles are produced in large quantities and have longer ranges than the primary ions, potentially triggering other reactions. Since we are only interested in the energy loss and fragmentation of the primary particles, electromagnetic particles were dumped on the spot, depositing their full energy without transport, saving a considerable amount of computational resources.\\
 \\
Simulation data was extracted from the simulations by requesting dedicated information such as particle type, position, direction and total energy or momentum each time a particle crosses from any region in the geometric model to the next one. A dummy region was created to score the quantities exactly at the DUT position. This was done either by compiling the FLUKA executable with a dedicated user routine called \emph{mgdraw.f}, or using the \emph{USRYIELD} card which scores a double-differential particle yield, generally calculated with respect to physical quantities particle charge, energy or LET. The formatted output could be easily post-processed for analysis.

\subsection{Beam properties from simulation}
This section of the report is divided in two parts: first, we focus on the primary beam properties such as energy straggling, the resulting beam energy and LET (subsection \ref{sec:primary}). Second, the secondary particle distribution present at the DUT position is discussed (subsection \ref{sec:distribution}). The simulations carried out served the purpose of preparation of the CHIMERA experimental campaign by selecting an adequate set of energies and LETs. \textbf{The simulation results will be expressed as function of the final set of (extracted) kinetic energies: 650 MeV/n, 750 MeV/n and 1 GeV/n}. Each of these also has an official operational "username": CPS.USER.EAST4, CPS.USER.EAST3 and CPS.USER.MD5.  The quantities discussed in this section are summarized in Table \ref{tab:summary}.

\subsubsection{Primary beam properties at DUT position}\label{sec:primary}

\begin{figure}[b!]
    \centering
    \includegraphics[width=\textwidth]{images/straggling.png}
    \vspace*{-0.5cm}
    \caption{FLUKA simulation results of the primary beam energy/momentum straggling for each of the energies used in the November 2022 experimental campaign, as function of the distance from the DUT position along the $s$-axis; shown for (a) the kinetic energy per nucleon, and (b) the momentum dispersion $\Delta P/P_0$ expressed in units of $[\%]$.}
    \label{fig:straggling_mean}
\end{figure}

As mentioned above, the primary Pb beam particles are sourced following a Gaussian momentum spread distribution with a $\pm 0.3\%$ standard deviation. E.g., at 1 GeV/n, a +0.3\% momentum offset amounts to a 5 MeV/n offset in kinetic energy per nucleon. Along the transfer line the momentum/energy distribution remains Gaussian, which for now we can disregard and focus on the mean. As shown in Fig. \ref{fig:straggling_mean}, it is clear that \textbf{the straggling is considerable and can be largely attributed to the IRRAD and CHARM experimental regions where the beam covers the largest distances in air}. The stopping power effect of the much denser materials present in the beam instrumentation and vacuum windows is obscured by the effect of the air since most of these elements are geometrically located inside these air sections (as also detailed in Table \ref{tab:material}). The kinetic energy per nucleon as function of distance from the DUT shown in Fig. \ref{fig:straggling} a) makes clear that the electronic stopping power due to the material budget has a qualitatively similar effect on all three primary energies in terms of energy loss with respect to the extracted primary energy. The 1 GeV/n beam is degraded down to 610 MeV/n, whereas the 650 MeV/n reaches down to 150 MeV/n. This has a profound effect on the associated LET which will be detailed further below. When expressed as momentum dispersion $\Delta P/P_0$ and shown in Fig. \ref{fig:straggling} b), it is clear that percentage-wise the lowest extracted energy is degraded the most: the 650 MeV/n kinetic energy beam loses over 50\% of its initial momentum.\\
\\
Given the long distance the beam travels through the transfer line and the presence of many non-vacuum sections, it is plausible that the primary beam energy/momentum spread is changed significantly. An initial Gaussian momentum distribution with a $\pm 0.3\%$ standard deviation may be spread out much more through successive interactions with materials. In what follows, we will express the distribution width in terms of the full width at half maximum, i.e. FWHM $\approx$ 2.355 $\sigma$. All primary beam energy distributions at the DUT are shown in Fig. \ref{fig:straggling}, including Gaussian fit parameters that show the mean and spread, the results are also summarized in Table \ref{tab:summary}. \textbf{For each extracted energy the FWHM of the distribution of primary beam particles is between 10 and 15 MeV/n}. This is not a big increase from the initial distribution spread which is also quoted in Table \ref{tab:summary}. \\
\\
For reference, we include the results of a simulation for 650 MeV/n extracted energy that is pencil-like; i.e., it has a fixed momentum and direction. The resulting distribution at the DUT position is indicative of a lower limit of spread which is inherent to the transfer line itself and amounts to slightly less than 1 MeV/n. The slightly lower mean value for the pencil beam is an effect of the beam line itself: the transmission of particles through the transfer line optics is affected by their particular momentum offset.

\begin{figure}[t!]
    \centering
    \includegraphics[width=\textwidth]{images/Resulting_energies.png}
    \vspace*{-0.5cm}
    \caption{FLUKA simulation results of the primary beam kinetic energy/momentum at the DUT position for each of the energies used in the November 2022 experimental campaign. Gaussian fit parameters are expressed in kinetic energy per nucleon [MeV/n]. The bottom axes express the beam parameters in terms of momentum and momentum offset. In addition, for 650 MeV/n the resulting distribution when using a pencil beam (fixed direction and momentum) is shown.}
    \label{fig:straggling}
\end{figure}


\begin{table}[h!]
   \scriptsize
    \centering
    \caption{Beam parameters subdivided between properties in the PS and at the DUT location. The properties in the PS also represent the FLUKA simulation initial conditions. For all quantities that were calculated from a distribution, the  mean and FWHM (in parentheses below) are quoted. The values for the pencil 650 MeV/n beam are only shown where applicable. From the simulations, we can assume a $\pm$ 1 MeVcm$^2$/mg error on the LET at the DUT location.}
    %\resizebox{\textwidth}{!}{%
    \label{tab:summary}
    \resizebox{\textwidth}{!}{%
    \begin{tabular}{c|rc|ccc||c|}
    \hline
    \hline
    & \multirow{2}{*}{USER} &&  CPS.USER & CPS.USER & CPS.USER & 650 MeV/n \\
    & && .EAST4 & .EAST3 & .MD5 & pencil \\
    \hline
    \multicolumn{1}{|c|}{\multirow{7}{*}{\rotatebox[origin=c]{90}{\makecell{EXTRACTED \\ FROM PS}}}} & & & & & & \\
    \multicolumn{1}{|c|}{} & Total momentum & & 351.92 & 291.15 & 265.79 & 265.79 \\
    \multicolumn{1}{|c|}{} & {[GeV/c]} &  & (2.47) & (2.04) & (1.86) & (0.0)  \\
    \multicolumn{1}{|c|}{} & & & & & & \\
    \multicolumn{1}{|c|}{} & Kinetic energy & & \textbf{1000.0} & \textbf{750.0} & \textbf{650.0} & 650.0 \\
     \multicolumn{1}{|c|}{}&  per nucleon [MeV/n]&  & (10.51) & (8.20) & (7.12) & (0.0)\\
      \multicolumn{1}{|c|}{} & & & & & & \\
      \hline
       \multicolumn{1}{c}{}\\
       \multicolumn{1}{c}{}\\
      \hline
    \multicolumn{1}{|c|}{\multirow{30}{*}{\rotatebox[origin=c]{90}{\makecell{AT CHARM \\ DUT LOCATION}}}} & & & & & & \\
        \multicolumn{1}{|c|}{} & Total momentum & & 255.46 & 170.31 & 116.69 & 115.54 \\
    \multicolumn{1}{|c|}{} & {[GeV/c]} & & (3.04) & (3.40) & (5.16) & (0.74) \\
    \multicolumn{1}{|c|}{} & & & & & & \\
    \multicolumn{1}{|c|}{} & Momentum offset & & 27.41 & 41.50 & 56.10 & 56.53 \\
    \multicolumn{1}{|c|}{} & $\Delta P/P_0$ [\%] & & (0.86) & (1.17) & (1.94) & (0.56) \\
    \multicolumn{1}{|c|}{} & & & & & & \\
    \multicolumn{1}{|c|}{} & Kinetic energy per nucleon & &\textbf{610.09} & \textbf{308.78} & \textbf{155.93} & 153.09\\
    \multicolumn{1}{|c|}{} & [MeV/n] & & (11.76) & (10.92) & (13.00) & (0.75) \\
    \multicolumn{1}{|c|}{} & & & & & &\\
    \multicolumn{1}{|c|}{} & & & & & &\\
    \multicolumn{1}{|c|}{} & \multirow{4}{*}{\makecell[r]{LET \\ {[MeVcm$^2$/mg]}}} & \multicolumn{1}{|c|}{\multirow{2}{*}{FLUKA}} & \textbf{13.42} & \textbf{18.40} & \textbf{26.53} & 26.83 \\
    \multicolumn{1}{|c|}{} & & \multicolumn{1}{|c|}{} & (0.10) & (0.32) & (1.33) & (0.19) \\
    \multicolumn{1}{|c|}{} & &\multicolumn{1}{|c|}{} & & & & \\
    \multicolumn{1}{|c|}{} & & \multicolumn{1}{|c|}{SRIM} & 13.43 & 17.77 & 25.67 & 25.95 \\
    \multicolumn{1}{|c|}{} & &  & & & & \\
    \multicolumn{1}{|c|}{} & &  & & & & \\
    \multicolumn{1}{|c|}{} & \multirow{3}{*}{\makecell[r]{Range (SRIM) \\ {[mm]}}} &\multicolumn{1}{|c|}{Si} & 28.73 & 10.12 & 3.95 & \\
    \multicolumn{1}{|c|}{} & & \multicolumn{1}{|c|}{}  & & & & \\
    \multicolumn{1}{|c|}{} & & \multicolumn{1}{|c|}{PMMA} & 45.71 & 16.36 & 5.61 &  \\
    \multicolumn{1}{|c|}{} & & & & & & \\
    \multicolumn{1}{|c|}{} & &  & & & & \\
    \multicolumn{1}{|c|}{} & \multirow{3}{*}{\makecell[r]{Energy deposition in \\300 $\mu$m of Si {[MeV]}}} &\multicolumn{1}{|c|}{FLUKA LET} & 938.06 & 1286.16 & 1854.45 & \\
    \multicolumn{1}{|c|}{} & & \multicolumn{1}{|c|}{}  & & & & \\
    \multicolumn{1}{|c|}{} & &\multicolumn{1}{|c|}{SRIM LET} & 938.76 & 1242.12 & 1794.33 & \\
    \multicolumn{1}{|c|}{} & & & & & & \\
    \multicolumn{1}{|c|}{} & &  & & & & \\
    \multicolumn{1}{|c|}{} & Fragmentation ratio & & \multirow{2}{*}{62.64} & \multirow{2}{*}{62.30} & \multirow{2}{*}{64.63} &\\
    \multicolumn{1}{|c|}{} & (Primary ions/total ions) [\%] & & & & &\\
    \multicolumn{1}{|c|}{} & &  & & & & \\
    \hline
    \hline
    
 \end{tabular}
    }
\end{table}


\subsubsection{Fragmentation and distributions at DUT positions}\label{sec:distribution}

As already introduced above, inelastic collisions of the primary beam particles with in-beam material inevitably causes the generation of secondary fragments. The large primary beam energy causes these fragments to travel along the beam and potentially arrive at the DUT testing location as well. This is visualised in Fig. \ref{fig:fragmentation} which shows that for all primary beams we get a continuous distribution in both energy per nucleon as atomic number Z. The distributions extend from zero up to the degraded primary beam energy (as shown in Fig. \ref{fig:straggling} and up to the Pb atomic number 82, showing that within these limits \textbf{we can expect a large variety of fragments at the DUT location}. For each of the extracted primary beams, a little over \textbf{a third of the particles arriving to the DUT location is a secondary fragment}. This is also summarized in Table \ref{tab:summary} as the ratio between primary beam ions and all ions arriving to the DUT.\\
\\
As result of these interactions, the particle energy, mass and charge can be altered which results in a different LET value. A full simulation taking into account both primary and secondary fragments is required to determine the LET distribution at the DUT location. This is shown in Fig. \ref{fig:LET} for each of the primary energies, where in each case a primary peak is accompanied by a tail of lower LET values. In the 1 GeV/n case, the modulation by the fragment atomic number is clearly visible: this is caused by the projectile-like fragments that are generated have roughly the same energy as the primary beam particle but have a different atomic number. This directly affects the electronic stopping power due to its Z-dependence (Eq. \ref{stopping}). The primary peaks were fitted with Gaussian functions to determine the centroid LET value and the width of the primary peak; the fit values are listed in Table \ref{tab:summary}. As shown above, the absolute spread of the primary kinetic energy distribution at the DUT is not drastically different between the three primary extracted values. \textbf{By consequence,  the LET spread is also limited, remaining well below 0.5 MeVcm$^2$/mg for the 1 GeV/n and 750 MeV/n extracted energies. At lower energy, also the LET variability is bigger, this is why the the LET spread at 650 MeV/n is almost 1.5 MeVcm$^2$/mg.} Another clear example is that the primary beam LET spread for the 650 MeV/n pencil-like beam is twice as large as the primary LET spread for 1 GeV/n. These are all important considerations to take into account when estimating the expected SEE response during testing.

\begin{figure}[t!]
    \centering
    \includegraphics[width=\textwidth]{images/fragmentation.png}
    \vspace*{-0.5cm}
    \caption{FLUKA simulation results of the beam fragmentation for each of the energies used in the November 2022 experimental campaign, showing the resulting beam kinetic energy per nucleon distribution at the DUT (left) and the atomic number Z of each of the fragments (right).}
    \label{fig:fragmentation}
\end{figure}

\begin{figure}[t!]
    \centering
    \includegraphics[width=\textwidth]{images/LETs.png}
    \vspace*{-0.5cm}
    \caption{FLUKA simulation results of the beam LET distribution at the DUT position for each of the energies used in the November 2022 experimental campaign. The primary peaks were fitted with a Gaussian to determine the parameters listed in Table \ref{tab:summary}. Note that in this plot the y-axis is in log scale, for this reason the FWHM of the primary peaks appears larger than it is.}
    \label{fig:LET}
\end{figure}

%%%%%%%%%%%%%%%%%%%%%%%%%%%%%%%%
%%%%%% Conclusion %%%%%%%%%%%%%%
%%%%%%%%%%%%%%%%%%%%%%%%%%%%%%%%
\clearpage
\clearpage
\section{Conclusion}
This report discussed the efforts to accurately manipulate and determine the beam energy and resulting LET of the beams used in CERN’s ongoing VHE heavy ion electronics testing activity, using FLUKA
simulations and solid state detector measurements. CHIMERA varies the primary beam energy by scaling the magnetic field in the Proton Synchrotron (PS) during acceleration and transport and uses a makerule that automatically scales the magnets in F61 and the T8 transfer line. This allows for seamless transition between three distinct beam energies in a matter of 2-3 minutes. Furthermore, with an automated energy scan script, these energy changes can occur at each new spill cycle, approximately every 15-30 seconds. The November 2022 run demonstrated the effectiveness of this system, with successful testing of four DUTs, a degrader scan, and measurements of three beam energies (650 MeV/u, 750 MeV/u, and 1000 MeV/u) at the end of the T08 transfer line which transports the beam from the PS to CHARM. The F61 and T08 line features a considerable amount of in-beam instrumentation and monitoring units, along a large fraction of distance traveled by the beam in air which compound to a total \SI{5.38}{\gram\per\centi\meter\squared}. This affects the beam quality through electronic stopping power, degrading the primary beam energy, and inelastic interactions causing the generation of secondary fragments. In addition, the scattering of beam particles causes the beam size to expand significantly. Based on surface density values of the different in-beam material regions, the biggest effects were expected in the different air sections. This was confirmed by detailed Monte Carlo (MC) calculations for which a detailed and accurate simulation model was constructed. The main FLUKA simulation results can be summarized as:
\begin{itemize}
    \item The primary beam energies are degraded significantly before reaching the CHARM testing location. Beams extracted at 1 GeV/n, 750 MeV/n and 650 MeV/n arrive at the DUT with energies 610, 309 and 156 MeV/n respectively.
    \item For each extracted energy the FWHM of the
distribution spread (FWHM) of primary beam particles is between 10 and 15 MeV/n. This is below 10\% from the mean degraded energy for all cases. The small energy spread spread at extraction is largely attributable to the slow extraction technique and the deliberate absence of a bunch rotation which aligns with the requirements of mono-energetic and mono-LET electronics testing
    \item We can expect a large
    variety of fragments at the DUT location. For each of the extracted primary beams,
    a little over a third of the particles arriving to the DUT location is a secondary
    fragment. These fragments do have a significantly lower LETs than the primary particle beam. From the lower abundance and lower LET of the fragments, it is expected that the primary beam dominates the SEE response.
    \item For the extracted beam energies at 1 GeV/n, 750 MeV/n and 650 MeV/n, the primary beam LET values at the DUT are 13.4, 18.4 and 26.5 MeVcm$^2$/mg respectively. Using either a lower extracted energy from the PS or degraders, the maximum theoretical LET for Pb can be reached at the DUT position (98.3 MeVcm$^2$/mg). This however goes at the expense of the particle range which drops below \SI{1}{mm} and is no longer compliant with SEE testing standards. 
    \item The primary beam LET spread is limited, remaining well below 0.5 MeVcm$^2$/mg for the 1 GeV/n and 750 MeV/n extracted energies. At lower energy, also the LET variability is bigger, this is why the the LET spread at 650 MeV/n is almost 1.5 MeVcm$^2$/mg which equals 5\% in relative terms.
    \item From simulations, the error on the LET is limited to maximum of $\pm$ 1 MeVcm$^2$/mg for the lowest energy beam. This is the conservative margin for all energies.
\end{itemize}
In addition, the particle energy could be estimated experimentally by taking energy deposition measurements with a silicon diode in the primary beam. As estimated by FLUKA and SRIM simulations, three distinct peaks in the energy deposition spectra should be observed, this was also confirmed experimentally. An energy scan, including a 21.6 mm PMMA plate directly upstream of the diode showed that we could determine the beam energy with an accuracy of 25 MeV/n. In the future, the simulation model can be readily updated for new beam configurations. Optimized experimental measurements at different locations in the beam line, using different degrader thicknesses and energy scans with improved resolution are envisaged. In conclusion, CHIMERA has provided an effective solution for primary beam energy control, offering the potential to explore a wide range of LET values with ease and precision. It is anticipated that this efficient and accurate method, made possible by careful preparation and automated processes, will set the standard for future ion runs.

\section{Timeline}

\begin{table}[H]
\caption{Timeline of solution implemented for the primary beam energy reduction.}
\centering
\begin{minipage}[t]{.7\linewidth}
\color{gray}
\rule{\linewidth}{1pt}
\ytl{2022}{Automatic scaling of magnets in transfer lines with energy}
\ytl{2022}{T8 transfer line convert to Pulse to Pulse Modulation (PPM)}
\ytl{Week 39}{Calculation of the PS main unit magnetic field required per beam energy}
\ytl{Week 40}{Stable beams at three different energies (1, 2 and 5.4 GeV/u)}
\ytl{Week 41}{First 750 MeV/u beam transmitted to MWPC}
\ytl{Week 42}{First 500 MeV/u beam. 400 MeV/u large and weak beam on East dump. Quadrupole scan on East Dump with 500 and 750 MeV/u beam}
\ytl{Week 43}{Dedicated Pb 750 MeV/u beam to CHARM}
\ytl{Week 45}{New beams with RFKO for ESA run: 650, 750 and 1000 MeV/u}
\ytl{Week 48}{ESA November Run}
\bigskip
\rule{\linewidth}{1pt}%
\end{minipage}%
\end{table}

\bibliography{references, reference_andreas}
\bibliographystyle{IEEEtran}

\end{document}
